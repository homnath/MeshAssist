\chapter{Getting started}
\label{chap:start}
\section{Package structure}
The \pack\ pacakge can be obtained using Git. Use the following command in the terminal:\\

\texttt{git clone \-\-recursive https://github.com/homnath/MeshAssist.git}\\

The package has the following structure:\\

\texttt{\pack/}
\begin{adescription}{~~CMakeLists.txt}
\item[~~LICENSE]               : License.
\item[~~Makefile]                : brief description of the package.
\item[~~bin/]                  : all object files and executables are stored in this folder.
\item[~~doc/]                  : documentation file/s for the \pack\ package.
\item[~~input/]                : contains input files.
\item[~~output/]               : default output folder. All output files are stored in this folder unless the different output path is defined from command.
\item[~~src/]                  : contains all source files.
\end{adescription}
 
\section{Prerequisites}

The package requires Make utility, latest C and Fortran compilers. For matlab files, Matlab is necessary.

\section{Configuration}

Open src/Makefile and modify the C and Fortran compilers if necessary.

\section{Compile}

Type the following command in the terminal

make all

Matlab files can be opened in and run from Matlab.

\section{Run}

{\em command} {\em input\_file} [{\em Options}]

Example:

./bin/xyz2jou ./input/xyz2jou\_example.utm

See Chapter "File Documentation" for all available commands. 

\section{Bug Report}

hgharti\_AT\_princeton\_DOT\_edu
